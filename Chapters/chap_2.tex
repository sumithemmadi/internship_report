\chapter{Tools and Technologies}\label{ch:tools-and-technologies}

\justify
{\myprojectname} encompasses a wide range of tools and technologies that facilitate the creation, deployment, and maintenance of applications.
From integrated development environments (IDEs) for coding to version control systems (VCS) for collaboration, programming languages, frameworks, database management systems (DBMS), cloud platforms, testing tools, API development tools, code editors, project management platforms, security tools, and monitoring/logging solutions.

\section{Tools}\label{sec:tools}
\justify

The following tools are used for the development, debugging, and testing of the application.

\begin{enumerate}[label=\roman*.]
  \item \textbf{Visual Studio Code}: A popular code editor developed by Microsoft, used to write and run code.\cite{VSCode}
  It is used to design the OSINT software, frontend, and backend of the application.
  \item \textbf{Flipper}: A desktop application developed by Facebook, used for debugging and checking the performance of the Android application. \cite{FLP}
  \item \textbf{React Native Debugger}: Used for debugging component rendering on different screens, including their structure, alignment, and styling~\cite{RDT}.
  \item \textbf{HTTPie}: A command-line HTTP client used for testing the APIs of the CallOne application~\cite{Httpie}.
  \item \textbf{Android Studio}: An IDE used to design the Android application in Kotlin.
  \item \textbf{Android Emulator}: A tool used to simulate Android devices on a computer to test the application on various devices and Android API levels~\cite{Android Studio}.
  \item \textbf{Git}: A version control system used for managing code among teams~\cite{Git}.
  \item \textbf{Chromium Browser}: Used to scrape various OSINT information of people from the internet.
  \item \textbf{DBeaver}: A database management tool used to connect, view, and edit the database of CallOne and OSINT software.
  \item \textbf{Nginx}: A web server used to run the backend server in production.
  \item \textbf{Puppeteer}: A Node library used to scrape data from webpages without detecting automation.
\end{enumerate}

\section{Technologies}\label{sec:technologies}
\justify

The following technologies are used for the development of the application.

\begin{enumerate}[label=\roman*.]
  \item \textbf{Kotlin}: A programming language used to build the Android application with Jetpack Compose~\cite{kt}.
  \item \textbf{Jetpack Compose}: A modern UI toolkit based on Material Design, used to create a responsive and visually appealing UI for the Android application~\cite{kt}.
  \item \textbf{TypeScript}: A superset of JavaScript that adds static types, used to catch errors early in the code editor~\cite{TypeScript}.
  \item \textbf{Express.js}: A web framework for Node.js, used to create the backend application~\cite{Express}.
  \item \textbf{Crypto-js}: An npm library used for encryption and decryption~\cite{Crypto-js}.
  \item \textbf{PostgreSQL}: A relational database management system used to store user data~\cite{psql}.
  \item \textbf{JSON Web Tokens (JWT)}: Used to authenticate users with the backend server, with each user receiving a unique token~\cite{JWT}.
  \item \textbf{PDFKit}: A library for generating PDF documents in Node.js and the browser~\cite{PDFKit}.
  \item \textbf{Baileys (WhatsApp Client)}: A WhatsApp client used to communicate with WhatsApp web sockets~\cite{Baileys}.
\end{enumerate}
